\documentclass{beamer}
\usetheme{CambridgeUS}
\usepackage{graphics}
\usepackage{graphicx}
%\usepackage{color}
\usepackage{latexsym,amsmath,amsfonts,amssymb}
\theoremstyle{plain}
\newtheorem{lem}{Lemma}[section]
\newtheorem{thm}[lem]{Theorem}
\newtheorem{prop}[lem]{Proposition}
\theoremstyle{definition}
\newtheorem{exa}[lem]{Example}
\newtheorem{rem}[lem]{Remark}
\newtheorem{obs}[lem]{Observation}
\newtheorem{ct}{Construction}
\newtheorem{proposition}[theorem]{Proposition}
\newtheorem{ps}{Proposition}
\newtheorem{cvs}{}
\newtheorem{defn}{Definition}[section]
\usepackage{multicol}
%\newcommand{\vl}{\color {violet}}
%\newcommand{\bl}{\color {blue}}
%\newcommand{\rd}{\color {red}}
%\newcommand{\yl}{\color {yellow}}
%\setbeamercolor{normal text}{bg=gray!40}
%\setbeamercolor{background canvas}{bg=violet!30}
\begin{document}
\baselineskip 12truept
\title{Python Basics}
%\subtitle{Research Proposal Presentation}
%\author{Dr. Vinayak Joshi\email{vvj@}}
\author[Hrishikesh Khaladkar,Fergusson College, Pune]{by\\Hrishikesh Khaladkar\\Department of Mathematics \\ Fergusson College, Pune}
%\institute{Department of Mathematics\\University of Pune\\Pune-411 007}\\
\vspace{0.5cm} 
%\date{$15^{th}$ Sept $2014$}
\frame{\titlepage}
%\section[Outline]{}
%\frame{\tableofcontents}
\section{Python Basics}
%\subsection{Prime ideals in Lattices}
\begin{frame}
      \frametitle{Python for Data Science}
      Python is widely used and very popular for a variety of software engineering tasks such as website development, cloud-architecture, back-end etc. It is equally popular in data science world. In advanced analytics world, there has been several debates on R vs. Python. There are some areas such as number of libraries for statistical analysis, where R wins over Python but Python is catching up very fast. With popularity of big data and data science, Python has become first programming language of data scientists.
       There are several reasons to learn Python. Some of them are as follows -
       \begin{enumerate}
      \item  Python runs well in automating various steps of a predictive model. 
       \item Python has awesome robust libraries for machine learning, natural language processing, deep learning, big data and artificial Intelligence. 
       \item Python wins over R when it comes to deploying machine learning models in production.
      \item It can be easily integrated with big data frameworks such as Spark and Hadoop.
      \item  Python has a great online community support.
    \end{enumerate}

\end{frame}

\begin{frame}
      \frametitle{Python for Data Science}
       Do you know these sites are developed in Python?
			\begin{itemize}
			\item YouTube
			\item Instagram
		   \item 	Reddit
			\item Dropbox
			\item Disqus
			\end{itemize}
			

\end{frame} 

\begin{frame}
      \frametitle{Basic Programs}
      \begin{enumerate}
      	\item Example1 \\
      	 x = 10 \\
      	 y = 3 \\
      	 print("10 divided by 3 is", x/y) \\
      	 print("remainder after 10 divided by 3 is", x\%y)
      	 \item Example 2 \\
      	  x = 100 \\
      	  x$ >$ 80 and x $<=$95 \\
      	  x$ >$ 35 or x$ <$  60 \\
      \end{enumerate}
      		
\end{frame}

\begin{frame}
      \frametitle{List of Arthimetic Operators}
      \begin{itemize}
     \item  + : Addition \\
     \item  - : Subtraction \\
      \item * : Multiplication \\
      \item / : Division \\
	  \item \% : Modulus (Remainder) \\
	 \item  ** : Power \\
	  \item // : Floor \\
	  \item (x+(d-1))//d : Ceiling \\
      \end{itemize}
\end{frame}


\begin{frame}
      \frametitle{Lists}
      Operations with lists
			\begin{itemize}
			\item Modify\ Replace an item from the list \\
			 X = [1, 2, 3] \\
             X[2]=5 \\
             print(X) (Expected Outcome: [1,2,5])
			
			\item Add\ Remove an item from the list\\
              X = ['AA', 'BB', 'CC'] \\
X.append('DD')\\
print(X) (Expected Output: ['AA', 'BB', 'CC', 'DD']) \\
 X = ['AA', 'BB', 'CC'] \\
X.remove('BB') \\
print(X) (Expected Output: ['AA', 'BB'])\\
		 \end{itemize}
			

\end{frame}


\begin{frame}
      \frametitle{Lists}
      Operations with lists
			\begin{itemize}
			\item Sort a List \\
			      k = [124, 225, 305, 246, 259]\\
    k.sort() \\
    print(k) (Expected Output : [124, 225, 246, 259, 305])


			
		 \end{itemize}
			

\end{frame}




\begin{frame}
      \frametitle{Tuples}
      Like list, tuple can also contain mixed data. But tuple cannot be changed or altered once created whereas list can be modified. Another difference is a tuple is created inside parentheses ( ). Whereas, list is created inside square brackets [ ]
	\begin{itemize}
			\item Initializing a Tuple \\
			 mytuple = (123,223,323) \\
            City = ('Delhi','Mumbai','Bangalore') \\
			\item Performig a loop on a Tuple\\
			 for i in City: \\
              print(i) \\
			(
			Expected Outcome :Delhi \\
			Mumbai \\
			Banglore \\
			)
		 \end{itemize}
			

\end{frame}

\begin{frame}
      \frametitle{Tuples}
      
	\begin{itemize}
			\item Tuples cannot be altered \\
			 X = (1, 2, 3) \\
             X[2]=5 \\
            ( TypeError: 'tuple' object does not support item assignment)             
             
			\end{itemize}
			

\end{frame}

\begin{frame}
      \frametitle{Dictionary}
      It works like an address book wherein you can find an address of a person by searching the name. In this example. name of a person is considered as key and address as value. It is important to note that the key must be unique while values may not be. Keys should not be duplicate because if it is a duplicate, you cannot find exact values associated with key. Keys can be of any data type such as strings, numbers, or tuples.
\begin{itemize}
			\item Create a dictionary \\
			     teams = $\lbrace$ 'Dave' : 'team A',\\
             'Tim' : 'team B',\\
             'Babita' : 'team C',\\
             'Sam' : 'team B',\\
             'Ravi' : 'team C' \\
            $\rbrace$
            \end{itemize}
			

\end{frame}

\begin{frame}
Operations with Dictionaries
\begin{itemize}
			\item Find Values \\
			     teams['Sam'] \\
			     (Expected Output : 'team B')
			 \item Delete an item \\
			      del teams['Ravi'] \\
			  \item Add an item \\
			       teams['Deep'] = 'team B'\\
			       (Expected output :
			       teams = $\lbrace$ 'Dave' : 'team A',\\
             'Tim' : 'team B',\\
             'Babita' : 'team C',\\
             'Sam' : 'team B',\\
             'Deep': 'team B'. \\
             'Ravi' : 'team C' \\
            $\rbrace$
			       )
            \end{itemize}
			

\end{frame}







\begin{frame}
 \frametitle{Sets}
      Sets are unordered collections of simple objects.
                        \begin{itemize}
                        \item How to define a set \\
                        X = set(['A', 'B', 'C']) \\
                        \item Does 'A' exist in X? \\
                        'A'in X \\
                        (Expected Output: True ) \\
                        \item Does 'A' exist in X? \\
                        'A'in X \\
                        (Expected Output: False ) \\
                        \item How to add 'D' in set X \\
                        X.add(D) \\
                        \item How to remove 'C' from set X?
                        X.remove(C) \\
                        \item How to create a copy of set X?
            Y=X.copy() \\
            \item Which items are common in both sets X and Y? Y \& X
     \end{itemize}
 \end{frame}
			
			
			
\frame{
\begin{center}
\Huge Thank You !!!
\end{center}  
 }
      
\end{document}
